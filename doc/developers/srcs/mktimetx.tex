
General notes

NVS

Note that this receiver uses UTC as the reference timescale to report time stamps.

This receiver reports a measurement time for observation a few hundred ms prior to the upcoming
second.

Trimble

 

Internals

One aspect to keep straight is that three timescales are used in the software:
PC time
UTC
GPS time

PC time is used only to match TIC and GNSS observations. Time stamps recorded in measurement files do
not have a fractional seconds part. The latency of the various signals (eg GPS messages 
for a second are always output after the beginning of the second) and their logging by the host PC 
means that this is no ambiguity during normal operation.

UTC time is used for CGGTTS generation.

GPS time is used for various calculations and for RINEX observations.

/subsection{Application.cpp}

Matched measurements are stored in a vector whose index corresponds to UTC time-of-day.

/subsection{CGGTTS.cpp}



/section{Adding support for a new receiver}

Conventions

A counter/timer measurement must be started by REF and stopped by GPS.
There is an option in gpscv.conf to reverse the sign of this.

The sawtooth correction is ADDED to the counter/timer measurement.


Debugging and validation

It can be useful to look at how well the receiver recovers GPS time - this is easily done by
using the option --disable-tic. The sawtooth-corrected TIC measurement is then set to zero.

REFSYS values noisy at the hundreds of ns level may indicate an off by one error in assigned time stamps.