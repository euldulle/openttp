\section{ublox receivers}

\subsection{ublox9log.py} \hypertarget{h:ublox9log}{}

This \cc{python3} script is used with ublox series 9 receivers like the ZED-F9P and ZED-F9T.
The following messages are enabled for 1 Hz output:
\begin{itemize}
	\item RXM-RAWX 
	\item TIM-TP
	\item NAV-SAT (not logged)
	\item NAV-TIMEUTC
	\item NAV-CLOCK
\end{itemize}

A status file is written once per minute, containing the SV identifiers of tracked satellites with
code and time synchronization flags set.

\subsubsection{usage}

\begin{lstlisting}[mathescape=true]
ublox9log.py [option] $\ldots$ 
\end{lstlisting}

The command line options are:
\begin{description*}
	\item[-{}-help, -h] print help and exit
	\item[-{}-config, -c \textless file\textgreater] use the specified configuration file
	\item[-{}-debug, -d] run in debugging mode
	\item[-{}-reset, -r]  reset the receiver before configuration
	\item[-{}-version, -v] print version information and exit
\end{description*}


\subsubsection{configuration}

\cc{ublox9log.pl} respects the \cc{receiver::model} configuration option.
The valid values are:
\begin{enumerate}
	\item ZED-F9
\end{enumerate}

\subsection{ubloxlog.pl} \hypertarget{h:ubloxlog}{}

This script used to configure and log series 8 receivers like the NEO-M8T.

\subsubsection{usage}

\begin{lstlisting}[mathescape=true]
ubloxlog.pl [option] $\ldots$ 
\end{lstlisting}

The command line options are:
\begin{description*}
	\item[-h] print help and exit
	\item[-c \textless file\textgreater] use the specified configuration file
	\item[-d] run in debugging mode
	\item[-r]  reset the receiver before configuration
	\item[-v] print version information and exit
\end{description*}

\subsection{ubloxextract.py} \hypertarget{h:ubloxextract}{}

\cc{ubloxextract.py} is used to decode and extract information from ublox receiver log files 
written in OpenTTP format. It will decompress the file if necessary.

It uses \cc{gpscv.conf} to construct receiver log file names if the file is not explicitly given.

\subsubsection{usage}

\begin{lstlisting}[mathescape=true]
ubloxextract.py [option] $\ldots$ [file]
\end{lstlisting}

Given an MJD via a command line option, it will construct a file name using the information in \cc{gpscv.conf}. If no MJD or file is given, it assumes the current MJD for the file. 

The command line options are:
\begin{description*}
	\item[-{}-help, -h] print help and exit
	\item[-{}-config, -c \textless file\textgreater] use the specified configuration file
	\item[-{}-debug, -d] run in debugging mode
	\item[-{}-version, -v] print version information and exit
	\item[-{}-mjd, -m \textless{MJD}\textgreater]  MJD of the file to process
  \item[-{}-uniqid]  show chip id
  \item[-{}-monver]  show hardware and software versions
  \item[-{}-navclock] extract nav-clock message
  \item[-{}-navsat] extract nav-sat message
  \item[-{}-navtimeutc] extract nav-timeutc message
  \item[-{}-rawx] extract raw measurement data
  \item[-{}-timtp] extract sawtooth correction
\end{description*}

\subsection{ubloxmkdev.py} \hypertarget{h:ubloxmkdev}{}

\cc{ubloxmkdev.py} is used for creating a device name for a connected 
receiver. It is useful, for example, when multiple receivers are connected
and unique USB device identification is not possible. It works by 
connecting to the receiver and querying its serial number. A configuration
file defines the device name for that serial number. 

The script is typically called by udev.

\subsubsection{usage}

\begin{lstlisting}[mathescape=true]
ubloxmkdev.py [option] $\ldots$ dev
\end{lstlisting}

The command line options are:
\begin{description*}
	\item[-{}-help, -h] print help and exit
	\item[-{}-debug, -d] run in debugging mode
	\item[-{}-version, -v] print version information and exit
\end{description*}

\subsubsection{configuration}

The configuration file is \cc{/usr/local/etc/ublox.conf}. It looks like:
\begin{lstlisting}
ce92fa1422 ublox1
77435b1763 ublox2
\end{lstlisting}
Each line consists of the receiver's serial number and the device name to be
associated with it.
