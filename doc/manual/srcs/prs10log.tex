\section{prs10log.pl}

\hypertarget{h:prs10log}{}

The Stanford PRS10 rubidium standard has a 1 pps input port that is typically used
to lock the PRS10 to a GNSS receiver. The PRS10 timetags each input 1 pps with respect to its own 1 pps and can report these measurements. It can thus be also be used as a time-interval counter. 
In this application, the lock to the input 1 pps is disabled 
(ie the PRS10 is left free-running),
and the 1 pps measurements are used for time-transfer.

\subsection{usage}

\begin{lstlisting}[mathescape=true]
prs10log.pl [option] $\ldots$ 
\end{lstlisting}

The command line options are:
\begin{description*}
	\item[-c \textless file\textgreater] use the specified configuration file
	\item[-d]	run in debugging mode
	\item[-h]	print help and exit
	\item[-v]	print version information and exit
\end{description*}

The PRS10 is also used as the system time reference so it has
entries in \cc{gpscv.conf} associated with this:
\begin{itemize}
	\item \hyperlink{h:reference_log_status}{reference::log status}
	\item \hyperlink{h:reference_logging_interval}{reference::logging interval}
	\item \hyperlink{h:reference_log_path}{reference::log path}
	\item \hyperlink{h:reference_file_extension}{reference::file extension}
	\item \hyperlink{h:reference_power_flag}{reference::power flag}
	\item \hyperlink{h:reference_status_file}{reference::status file}
\end{itemize}
