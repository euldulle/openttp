\subsection{editcggtts.py}

\hypertarget{h:editcggtts}{}

\cc{editcggtts.py} is used to edit CGGTTS navigation files. 
The header checksum and the modification time are updated after editing, 

\subsubsection{usage}

\begin{lstlisting}[mathescape=true]
editcggtts.py [OPTION] $\ldots$ filename [filename $\ldots$]
\end{lstlisting}
The command line options are:
\begin{description*}
	\item[-{}-help,-h]	print the help information and exit
	\item[-{}-debug,-d]	print debugging information to stderr
	\item[-{}-comments COMMENTS]   set comment
  \item[-{}-output OUTPUT, -o OUTPUT] output to file/directory OUTPUT
  \item[-{}-tmp]    output is to file(s) with .tmp added to name
  \item[-{}-replace, -r]  replace the edited file(s)
  \item[-{}-nosequence]  do not interpret (two) input file names as a sequence
  \item[-{}-nowarn]  suppress warnings
	\item[--version,-v] show the version information and exit
\end{description*}

The \cc{--output}, \cc{--tmp} and \cc{--replace} options are mutually exclusive. 
If none of these options are used, output will be to \cc{stdout}.

If two filenames are given without the \cc{--nosequence} option, they will be
interpreted as specifying a sequence of files to be processed.
