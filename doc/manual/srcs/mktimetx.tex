
\section{mktimetx}

\hypertarget{h:mktimetx}{}

\cc{mktimetx} is the core OpenTTP application.
It creates CGGTTS and RINEX-format time-transfer files.

In the RINEX files, the code measurements have been corrected for any offsets between the raw measurements
and the output 1 pps, and then the correction to the external clock applied ie the raw code measurements
are with respect to the external clock.

\subsection{usage}

The command line options are
\begin{description*}
	\item[--configuration \textless file \textgreater] 
	\item[--counterpath \textless path \textgreater]	
	\item[--comment \textless string>] comment for the CGGTTS file	
	\item[--debug \textless file \textgreater]	turn on debugging to \cc{file}. To debug to \cc{stderr}, just use `stderr'.
	\item[--disable-tic] disable the use of counter/timer measurements in 
	\item[--help] show help
	\item[--no-navigation] disable output of a RINEX navigation file
	\item[--receiver-path \textless path \textgreater]
	\item[--short-debug-message]
	\item[--sv-diagnostics] save raw measurements for each SV in a file. Each SV is in a separate file.
	\item[--timing-diagnostics] save timing diagnostics in a file
	\item[--verbosity \textless 1-4 \textgreater] set the debugging verbosity
	\item[--version] print version information	
\end{description*}

\subsection{configuration file}

\cc{mktimtex} uses \cc{gpscv.conf}.
Keys used by it are listed in table \ref{t:gpscvKeys}.

\begin{table}
\begin{tabular}{l|p{10cm}}
Section & Key \\ \hline
Antenna & antenna number, antenna type, \textit{delta H}, \textit{delta N},
         \textit{delta E}, frame, marker name, marker number, marker type, 
         x, y, z \\ \hline
CGGTTS  & \textit{comments}, \textit{create}, lab, lab id, 
         \textit{maximum dsg}, \textit{minimum track length},\textit{naming convention},
         outputs, receiver id, reference, revision date, version\\
Counter & \textit{file extension}, \textit{flip sign}\\ \hline
Delays  &  antenna cable, reference cable\\
Misc & \textit{gzip}\\
Paths & cggtts, counter data, receiver data, \textit{processing log},
        rinex, \textit{root}, tmp\\
Receiver & \textit{file extension}, manufacturer, model,
          \textit{observations}, pps offset, \textit{sawtooth phase},
          \textit{version}\\ \hline
RINEX & agency, \textit{create}, observer, version\\
\end{tabular}
\caption{Summary of \cc{gpscv.conf} entries used by \cc{mktimetx}. Optional entries are italicised. \label{t:gpscvKeys}}
\end{table}

\begin{lstlisting}




\end{lstlisting}

\subsection{log file}