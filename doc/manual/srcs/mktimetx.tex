
\section{mktimetx}

\hypertarget{h:mktimetx}{}

\cc{mktimetx} is the core OpenTTP application.
It creates CGGTTS and RINEX-format time-transfer files.

In the RINEX files, the code measurements have been corrected for any offsets between the raw measurements
and the output 1 pps, and then the difference between the external clock and output pps (obtained from the TIC measurements)
is applied ie the raw code measurements are reported with respect to the external clock.

\subsection{usage}

When run with no arguments, \cc{mktimetx} uses the default GPSCV processing configuration file 
\cc{/home/cvgps/etc/gpscv.conf} and processes data from the preceding day.

The command line options are
\begin{description*}
	\item[-{}-configuration \textless file\textgreater] specify the configuration file
	\item[-{}-counterpath \textless path\textgreater]	specify the path to the counter/timer measurements
	\item[-{}-comment \textless string\textgreater] comment for the CGGTTS file	
	\item[-{}-debug \textless file\textgreater]	turn on debugging to \cc{file}. To debug to \cc{stderr}, just use `stderr'.
	\item[-{}-disable-tic] disable the use of sawtooth-corrected counter/timer measurements 
	\item[-{}-help] show help
	\item[-m \textless MJD\textgreater] specify the mjd
	\item[-{}-no-navigation] disable output of a RINEX navigation file
	\item[-{}-receiver-path \textless path\textgreater] specify the path to the GNSS raw data
	\item[-{}-short-debug-message] print out shorter debugging messages
	\item[-{}-sv-diagnostics] save raw measurements for each SV in a file. Each SV is in a separate file.
	\item[-{}-timing-diagnostics] save timing diagnostics in a file
	\item[-{}-verbosity \textless 1-4\textgreater] set the debugging verbosity
	\item[-{}-version] print version information	and exit
\end{description*}
Example:
\begin{lstlisting}
mktimetx --configuration test.conf -m 57803 --debug stderr --verbosity 1 
\end{lstlisting}
runs \cc{mktimetx} in debugging mode, writing to \cc{stderr} using the configuration file \cc{test.conf} and processing
data for MJD 57803.

\subsection{configuration file}

\cc{mktimtex} uses \cc{gpscv.conf}.
Keys used by it are listed in table \ref{t:gpscvKeys}.

\begin{table}
\begin{tabular}{l|p{10cm}}
Section & Key \\ \hline
Antenna & antenna number, antenna type, \textit{delta H}, \textit{delta N},
         \textit{delta E}, frame, marker name, marker number, marker type, 
         x, y, z \\ \hline
CGGTTS  & \textit{comments}, \textit{create}, lab, lab id, 
         \textit{maximum dsg}, \textit{minimum track length},\textit{naming convention},
         outputs, receiver id, reference, revision date, version\\
Counter & \textit{file extension}, \textit{flip sign}\\ \hline
Delays  &  antenna cable, reference cable\\
Misc & \textit{gzip}\\
Paths & cggtts, counter data, receiver data, \textit{processing log},
        rinex, \textit{root}, tmp\\
Receiver & \textit{file extension}, manufacturer, model,
          \textit{observations}, pps offset, \textit{sawtooth phase},
          \textit{version}\\ \hline
RINEX & agency, \textit{create}, observer, version\\
\end{tabular}
\caption{Summary of \cc{gpscv.conf} entries used by \cc{mktimetx}. Optional entries are italicised. \label{t:gpscvKeys}}
\end{table}

\begin{lstlisting}




\end{lstlisting}

\subsection{log file}