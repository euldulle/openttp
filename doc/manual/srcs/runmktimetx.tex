
\section{runmktimetx.pl \label{runmktimetx}}

\hypertarget{h:runmktimetx}{}

\cc{runmktimetx.pl} provides a convenient way to process multiple days of data and to run any missed processing.

\cc{runmktimetx.pl} uses \cc{gpscv.conf}. There are no entries in \cc{gpscv.conf} specific to \cc{runmktimetx.pl}

\cc{runmktimetx.pl} doesn't produce a log file.
	
\subsection{usage}
\cc{runmktimetx.pl} is normally run as a \cc{cron} job.

To run \cc{runmktimetx.pl} on the command line, use
\begin{lstlisting}[mathescape=true]
runmktimetx.pl [option] $\ldots$ [Start MJD  [Stop MJD]]
\end{lstlisting}

\cc{Start MJD} and \cc{Stop MJD} specify the range of MJDs to process.
If a single MJD is specified, then data for that day is processed. If no
MJD is specified, the previous day's data is processed.

The options are:
\begin{description*}
	\item[-a \textless{file}\textgreater]  extend check for missed processing back \cc{n} days 
		(the default is~7)
	\item[-c \textless{file}\textgreater] use the specified configuration file
	\item[-d]	run in debugging mode
	\item[-h]	print help and exit
	\item[-x] run missed processing
	\item[-v]	print version information and exit
\end{description*}
