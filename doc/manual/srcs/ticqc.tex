\subsection{ticqc.py}

\hypertarget{h:ticqc}{}

\cc{ticqc} checks TIC files. It currently reports the total number of measurements, duplicates and gaps,
as well the data range.

\subsubsection{usage}

\begin{lstlisting}[mathescape=true]
ticqc.py [option] infile
\end{lstlisting}
The command line options are:
\begin{description*}
\item[-{}-help, -h] print help and exit
\item[-{}-verbose ] verbose output eg duplicate measurements are printed out
\item[-{}-version, -v] print version information and exit
\end{description*}

