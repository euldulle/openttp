
\section{okcounterd \label{sokcounterd}}

\hypertarget{h:okcounterd}{}

\cc{okcounterd} provides the interface to the Opal Kelly FPGA development board when configured as a multi-channel counter.
It communicates with user processes via port 21577 (it's the date ``Star Wars'' premiered).
Mutiple processes can communicate with the daemon so that logging processes can be separated.

\cc{okcounterd} does not use a configuration file and does not produce a log file.

\cc{okcounterd} recognizes the following commands, sent as plain text:
\begin{description*}
	\item[] CONFIGURE GPIO $0\vert1$ enables/disables the system GPIO.
	\item[] CONFIGURE PPSSOURCE n selects the input channel of the counter which is
	routed to the output 1 pps. 
	\item[] QUERY CONFIGURATION reads the device configuration register. \cc{okcounterd} sends
	a plain text response.
	\item[] LISTEN registers a process to receive counter-timer readings.
\end{description*}
\cc{okcounterdctrl.pl} provides a convenient way to send commands.

Counter/timer data sent by \cc{okcounterd} is in the following format:
\begin{lstlisting}[mathescape=true]
channel_number  timestamp (s) timestamp ($\mu$s) reading (ns)
\end{lstlisting}
Channel numbers are indexed from 1.

\subsection{usage}
\cc{okcounterd} is automatically started by the system's init system. On Debian, this is \cc{systemd}. 
It can be run manually for debugging purposes. Use:
\begin{lstlisting}[mathescape=true]
okcounterd [OPTION] $\ldots$
\end{lstlisting}
The command line options are:
\begin{description*}
	\item[-b] \textless{file}\textgreater load the specified bitfile (the full path is needed)
	\item[-d]	run in debugging mode
	\item[-h]	print help and exit
	\item[-v]	print version information and exit
\end{description*}
To manually run \cc{okcounterd}, you may need to disable the system service
and kill any running \cc{okcounterd} process.
