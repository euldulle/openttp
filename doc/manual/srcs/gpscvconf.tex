\section{gpscv.conf - the core configuration file \label{sgpscvconf} }

A single configuration file, \cc{gpscv.conf}, provides configuration information to most of the
OpenTTP software. 
\cc{gpscv.conf} is used by \cc{mktimetx}, receiver logging scripts, TIC logging scripts,receiver utilities and so on.

It uses the format described in \ref{sConfigFileFormat}.

\subsection{[CGGTTS] section}

Entries in this section control the format and content of CGGTTS files and filtering applied to CGGTTS tracks.

{\bfseries BIPM cal id}\\
This defines CAL\_ID for the internal delay, as used in v2E CGGTTS headers.\\
\textit{Example:}
\begin{lstlisting}
BIPM cal id=none
\end{lstlisting}
{\bfseries comments}\\
This defines COMMENTS in the CGGTTS header.\\
\textit{Example:}
\begin{lstlisting}
comments=none
\end{lstlisting}
{\bfseries create}\\
This defines whether or not CGGTTS files will be generated.\\
\textit{Example:}
\begin{lstlisting}
create=yes
\end{lstlisting}
{\bfseries comments}\\
This defines COMMENTS in the CGGTTS header.\\
\textit{Example:}
\begin{lstlisting}
comments=none
\end{lstlisting}
{\bfseries ephemeris}\\
This defines whether to use the receiver-provided ephemeris or a user-provided ephemeris (via a RINEX navigation file).
If a user-provided ephemeris is specified then \cc{ephemeris path} and \cc{ephemeris file} 
must also be specified.\\
\textit{Example:}
\begin{lstlisting}
ephemeris=receiver
\end{lstlisting}
{\bfseries ephemeris file}\\
This specifies a pattern for user-provided RINEX navigation files.
Currently, only patterns of the form \cc{XXXXddd0.yyn} are recognized.\\
\textit{Example:}
\begin{lstlisting}
ephemeris file=SYDNddd0.yyn
\end{lstlisting}
{\bfseries ephemeris path}\\
This specifies the path to user-provided RINEX navigation files.\\
\textit{Example:}
\begin{lstlisting}
ephemeris path=igsproducts
\end{lstlisting}
{\bfseries internal delay}\\
This defines INT DLY in the CGGTTS header. The units are ns.\\
\textit{Example:}
\begin{lstlisting}
INT DLY=0.0
\end{lstlisting}
{\bfseries lab id}\\
This defines the two-character lab code used for creating BIPM-style file names, as per the V2E specification.\\
\textit{Example:}
\begin{lstlisting}
lab id=AU
\end{lstlisting}
{\bfseries maximum DSG}\\
CGGTTS tracks with DSG lower than this will be filtered out. 
The units are ns.\\
\textit{Example:}
\begin{lstlisting}
maximum DSG = 10.0
\end{lstlisting}
{\bfseries minimum elevation}\\
CGGTTS tracks lower than this will be filtered out. 
The units are degrees.\\
\textit{Example:}
\begin{lstlisting}
minimum elevation = 10
\end{lstlisting}
{\bfseries minimum track length}\\
CGGTTS tracks shorter than this will be filtered out. Tracks meeting the criterion are not necessarily contiguous.
The units are seconds.\\
\textit{Example:}
\begin{lstlisting}
minimum track length = 390
\end{lstlisting}
{\bfseries naming convention}\\
Defines the CGGTTS file naming convention. Valid options are `plain' (MJD.cctf) and `BIPM'.
The \cc{lab id} and \cc{receiver id} should be defined in conjunction with BIPM-style filenames.\\
\textit{Example:}
\begin{lstlisting}
naming convention = BIPM
\end{lstlisting}
{\bfseries outputs}\\
This defines a list of sections which in turn define the desired CGGTTS outputs.\\
\textit{Example:}
\begin{lstlisting}
outputs=CGGTTS-GPS-C1,CGGTTS-GPS-P1,CGGTTS-GPS-P2
\end{lstlisting}
{\bfseries reference}\\
This defines REF in the CGGTTS header.\\
\textit{Example:}
\begin{lstlisting}
reference=UTC(XXX)
\end{lstlisting}
{\bfseries receiver id}\\
This defines the two-character receiver code used for creating BIPM-style file names, 
as per the V2E specification.\\
\textit{Example:}
\begin{lstlisting}
receiver is=01
\end{lstlisting}
{\bfseries revision date}\\
This defines REV DATE in the CGGTTS header. It must be in the format YYYY-MM-DD.\\
\textit{Example:}
\begin{lstlisting}
revision date = 2015-12-31
\end{lstlisting}
{\bfseries version}\\
This defines the version of CGGTTS output.Valid versions are v1 and v2E. 
The \cc{lab id} and \cc{receiver id} should be defined in conjunction with v2E ouput\\
\textit{Example:}
\begin{lstlisting}
version = v2E
\end{lstlisting}

\subsubsection{CGGTTS output sections}

Multiple CGGTTS outputs can be defined, allowing for different constellation and signal combinations.
An example of a CGGTTS output section is as follows:
\begin{lstlisting}
[CGGTTS-GPS-C1]
constellation=GPS
code=C1
path=cggtts
BIPM cal id = none
internal delay = 11.0
\end{lstlisting}

The new entries for a CGGTTS output section are:\\
{\bfseries code}\\
This defines the GNSS signal code. Valid values are C1,P1 and P2.\\
\textit{Example:}
\begin{lstlisting}
code=C1
\end{lstlisting}
{\bfseries constellation}\\
This defines the GNSS constellation. Only GPS is supported currently.\\
\textit{Example:}
\begin{lstlisting}
constellation=GPS
\end{lstlisting}
{\bfseries path}\\
This defines the directory in which output files are placed.\\
\textit{Example:}
\begin{lstlisting}
path=cggtts
\end{lstlisting}


\subsection{[RINEX] section}

Entries in this section control the format and content of RINEX files.

{\bfseries agency}\\
This specifies the value of the AGENCY field which appears in RINEX observation file headers.\\
\textit{Example:}
\begin{lstlisting}
agency=MY AGENCY
\end{lstlisting}
{\bfseries create}\\
This defines whether or not RINEX files will be generated.\\
\textit{Example:}
\begin{lstlisting}
create = yes
\end{lstlisting}
{\bfseries users}\\
This specifies the value of the OBSERVER field which appears in RINEX observation file headers.
If the observer is specified as `user' then the environment variable USER is used.\\
\textit{Example:}
\begin{lstlisting}
observer=user
\end{lstlisting}
{\bfseries version}\\
This  specifies the version of the RINEX output. Valid versions are 2 and 3.\\
\textit{Example:}
\begin{lstlisting}
version=2
\end{lstlisting}


