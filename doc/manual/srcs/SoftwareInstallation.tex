%%
%% \chapter{Software installation}
%% 

\section{Installation requirements}

You will need a basic Linux development environment, including C/C++ compilers, Perl and python2 and python3.
You will likely need the development packages for:
\begin{description*}
	\item[\cc{boost}]  C++ libraries
	\item[\cc{libgsl}] GNU scientific library
\end{description*}

You may also need:
\begin{description*}
	\item[\cc{Time::HiRes}] Perl library
\end{description*}

The above requirements are by no means comprehensive.
What you will need to install will depend very much on the
Linux distribution you are using.

\section{Obtaining, building and installing the software}

The OpenTTP software is obtained from the git repository hosted by GitHub.
There are two main branches, the `master' and `develop' branches.
The `develop' branch is generally stable, but may occasionally be broken.

Clone the repository
\begin{lstlisting}
git clone https://github.com/openttp/openttp.git
\end{lstlisting}
change directory to the repository and then check out the branch you want to use
\begin{lstlisting}
git checkout develop
\end{lstlisting}

\begin{figure}
\dirtree{%
	.1 openttp.
	.2 doc.
	.2 software.
	.3 gpscv.
	.4 common.
	.5 bin.
	.5 etc.
	.5 process.
	.6 mktimetx.
	.5 prs10c.
	.4 gpsdo.
	.4 javad.
	.4 nvs.
	.4 trimble.
	.4 ublox.
	.3 system.
	.4 device-tree-overlays.
	.4 src.
	.5 dioctrl.
	.5 fpga.
	.5 lcdmon.
	.5 libconfigurator.
	.5 ntpd.
	.5 okbitloader.
	.5 okcounterd.
	.5 OpenOK2.
	.5 ppsd.
	.5 sysmonitor.
	.4 udev.
}
\caption{Overview of the OpenTTP software source}
\end{figure}

\subsection{Building the documentation}

The documentation is written in LaTeX and uses some packages and fonts that are generally packaged as `extras'
and will probably need to be installed. The various packages used can be inspected in the file
\cc{doc/manual/OpenTTPManual.tex}.
To build the documentation as a PDF, change directory to \cc{doc/manual} and run \cc{make}.

\subsection{Installing the software}

Two scripts are provided for building and installing the software.
\begin{lstlisting}
software/system/installsys.pl
software/gpscv/install.pl
\end{lstlisting}

The installer has been used with current versions of RedHat Enterprise Linux, 
CentOS, Ubuntu and Debian (on the BeagleBone Black).

\cc{installsys.pl} must be run first (with superuser privileges), 
because it installs libraries which are needed by the GPSCV software.

\cc{installsys.pl} can also be used to install various targets individually.
Run \cc{installsys.pl -h} to see the options available. 
This may be helpful when trying to resolve problems with building the software.

\cc{install.pl} will archive any existing scripts and binaries, and create any
directories it needs in the current user's home directory. 
Run \cc{install.pl -h} to see the options available.

\section{A minimal software setup}

For users who wish to use their own hardware,
this section describes the minimum setup required for operation.

The OpenTTP software distribution is comprised of various C/C++ applications and Perl and python scripts.
As a minimum, you will need:
\begin{description*}
	\item \cc{TFLibrary.pm}, a Perl module for commonly used tasks, such as reading configuration files
	\item \cc{ottplib.py}, the \cc{python} version of \cc{TFLibrary.pm}
	\item \cc{libconfigurator}, a C library for parsing configuration files
	\item \cc{lockport}, a utility to create a UUCP lock file
	\item \cc{mktimetx}, to create time-transfer files
	\item one of the OpenTTP-provided scripts to log your receiver
	\item one of the OpenTTP-provided scripts to log your counter/timer
\end{description*}

You must use one of the OpenTTP receiver logging scripts with your GNSS receiver because \cc{mktimetx}
expects a custom file-format \ref{s:DataFileFormat}. In particular, the
receiver's native binary formats are not readable by \cc{mktimetx}. Similarly, OpenTTP uses a custom file
format for the counter-timer measurement files, although in this case, conversion from another format
will probably be straight-forward. Most likely, users will have to provide their own software for
logging their counter/timer, given the large number of possible devices here, and the limited
support within OpenTTP.

You may also find the following useful:
\begin{description*}
	\item[] \cc{kickstart.pl} for automatic start of logging processes
	\item[] \cc{runmktimetx.pl} for automated processing, and reprocessing
	\item[] \cc{gziplogs.pl} for managing log file compression
\end{description*}


\section{Common configuration problems}

\subsection{UUCP lock file creation}

UUCP lock files are used to prevent non-OTTP processes from opening serial ports while they are 
in use. You need to set the location for lock file creation specific to the operating system, 
and ensure that the OTTP user (cvgps typically) has the right permissions to write to this location. 
This is typically achieved by adding the user to the appropriate group. 
For example, on Ubuntu 18 the lock directory is \cc{/var/lock}. 
In this case, the directory is world writeable, so 
the OTTP user does not need to belong to any supplementary groups.
