\section{lcdmonitor}

\cc{lcdmonitor} runs the lcd display on the front panel of the  unit.

\subsection{usage}
\cc{lcdmonitor} is automatically started by the system's init system.
On Debian, this is \cc{systemd}. It can be manually for debugging purposes.

The command line options are
\begin{description*}
	\item[-d]	run in debugging mode
	\item[-h]	print help and exit
	\item[-v]	print version information and exit
\end{description*}
To run \cc{ldcmonitor} on the command line, you will need to disable the entry in \cc{/etc/inittab},
reread the \cc{inittab} and kill any running \cc{lcdmonitor} process.

\subsection{configuration file \label{confformat}}

The configuration file for \cc{ldcmonitor} is \cc{/usr/local/etc/lcdmonitor.conf}. This file is
only modifiable by the super-user. The file is divided into
sections, with section names delimited by square brackets [\space]. Entries in each section
are of the form:
\begin{lstlisting}
token = value
\end{lstlisting}
Comments begin with a \cc{\#} character. Entries in the various sections of the configuration file
are given below. 

\subsubsection{[General] section}

{\bfseries NTP user}\\
This  defines the name of the user associated with NTP functions.
The entry in the configuration file looks like:
\begin{lstlisting}
NTP user = ntp-admin
\end{lstlisting}
{\bfseries Squealer config}\\
This  specifies the location of the configuration file used by \cc{squealer}, a
program used to detect system problems.
The entry in the configuration file looks like:
\begin{lstlisting}
Squealer config = /home/cvgps/etc/squealer.conf
\end{lstlisting}

\subsubsection{[UI] section}
{\bfseries Show PRNs}\\
This specifies whether or not to show the PRNs (or Space Vehicle identifiers) of
GPS satellites being tracked by the receiver. If this is set to zero, then only
the number of satellites tracked is displayed.
The entry in the configuration file looks like:
\begin{lstlisting}
Show PRNs=1
\end{lstlisting}
{\bfseries LCD intensity}\\
This sets the intensity of the LCD display. Valid values are 0 to 100.
The entry in the configuration file looks like:
\begin{lstlisting}
LCD intensity=90
\end{lstlisting}
{\bfseries LCD contrast}\\
This sets the contrast of the LCD display. Valid values are 0 to 100.
The entry in the configuration file looks like:
\begin{lstlisting}
LCD contrast=95
\end{lstlisting}

\subsubsection{[GPSCV] section}
{\bfseries GPSCV user}\\
This  defines the name of the user associated with GPSCV functions.
The entry in the configuration file looks like:
\begin{lstlisting}
GPSCV user = cvgps
\end{lstlisting}
{\bfseries gpscv config}\\
This defines the location of the file \cc{gpscv.conf}.
The entry in the configuration file looks like:
\begin{lstlisting}
GPSCV config = /home/cvgps/etc/gpscv.conf
\end{lstlisting}
{\bfseries GPS restart command}\\
This specifies the command used to restart the GPS receiver. Note that since
\cc{lcdmonitor} runs as \cc{root}, the restart must be explicitly done
as \cc{cvgps}.
The entry in the configuration file looks like:
\begin{lstlisting}
GPS restart command =su - cvgps -c '/home/cvgps/bin/check_rx'
\end{lstlisting}
{\bfseries GPS logger lock file}\\
This specifies location of the lock file used by the GPS logging process. It is
used to determine which process needs to be killed before a restart.
The entry in the configuration file looks like:
\begin{lstlisting}
GPS logger lock file=/home/cvgps/logs/rx.lock
\end{lstlisting}

\subsubsection{[OS] section}
{\bfseries Reboot command}\\
This specifies the command used to reboot the PC.
The entry in the configuration file looks like:
\begin{lstlisting}
Reboot command = /sbin/shutdown -r now
\end{lstlisting}
{\bfseries Poweroff command}\\
This specifies the command used to shut down the PC.
The entry in the configuration file looks like:
\begin{lstlisting}
Poweroff command = /sbin/shutdown -t 3 -h
\end{lstlisting}
{\bfseries ntpd restart command}\\
This specifies the command used to restart \cc{ntpd}.
The entry in the configuration file looks like:
\begin{lstlisting}
ntpd restart command = /sbin/service ntpd restart
\end{lstlisting}

\subsubsection{[Network] section}
{\bfseries DNS}\\
The entry in the configuration file looks like:
\begin{lstlisting}
DNS = /etc/resolv.conf
\end{lstlisting}
{\bfseries Network}\\
The entry in the configuration file looks like:
\begin{lstlisting}
Network = /etc/sysconfig/network
\end{lstlisting}
{\bfseries Eth0}\\
The entry in the configuration file looks like:
\begin{lstlisting}
Eth0 = /etc/sysconfig/network-scripts/ifcfg-eth0
\end{lstlisting}

\subsection{log files}
\cc{lcdmonitor} produces a log file \cc{/usr/local/log/lcdmonitor.log} that records any
actions that there made from the front panel, and a lock file \cc{/usr/local/log/lcdmonitor.lock}
that is used to prevent duplicate processes from running.