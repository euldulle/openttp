\section{Trimble Resolution T receivers}

Trimble Resolution T receivers are obsolete and further development of the software
described here has ceased. The successor to the Resolution T, the SMT 360, 
cannot currently be used for time-transfer because of changes in the message set.  
It is possible though that a future TSIP-based receiver may be suitable for time-transfer.

\subsection{restlog.pl} \hypertarget{h:restlog}{}

\subsubsection{usage}

\begin{lstlisting}[mathescape=true]
restlog.pl [option] $\ldots$ 
\end{lstlisting}

The command line options are:
\begin{description*}
	\item[-c \textless{file}\textgreater] use the specified configuration file
	\item[-h] print help and exit
	\item[-d] run in debugging mode
	\item[-r] reset the receiver on startup
	\item[-v] print version information and exit
\end{description*}

\subsubsection{configuration}

\cc{restlog.pl} respects the \cc{receiver::model} configuration option.
The valid values are:
\begin{enumerate}
	\item Resolution T
	\item Resolution SMT 360 
\end{enumerate}

\subsection{restextract.pl} \hypertarget{h:restextract}{}

\cc{restextract.pl} is used to decode and extract information from Trimble receiver log files. 
It will decompress the file if necessary.

\subsubsection{usage}

\begin{lstlisting}[mathescape=true]
restextract.pl [option] $\ldots$ 
\end{lstlisting}

Given an MJD via a command line option, it will construct a file name using the information in \cc{gpscv.conf}. If no MJD is given, it assumes the current MJD for the file.

The command line options are:
\begin{description*}
	\item[-c \textless{file}\textgreater] use the specified configuration file
	\item[-h] print help and exit
	\item[-v] print version information and exit
  \item[-a] extract S/N for visible satellites 
	\item[-f] show firmware version
  \item[-l] leap second warning
  \item[-L] leap second info )
  \item[-m \textless{mjd}\textgreater]  MJD of the file to process
  \item[-p] position fix message
  \item[-r \textless{svn}\textgreater] pseudoranges for satellite with given SVN 
			(svn=999 reports all satellites)
  \item[-s] number of visible satellites
  \item[-t] temperature 
  \item[-u] UTC offset 
  \item[-x] alarms and gps decoding status
\end{description*}

\subsection{restinfo.pl} \hypertarget{h:restinfo}{}

\cc{restinfo.pl} communicates with the receiver configured in \cc{gpscv.conf},
polling it for information such as hardware, software and firmware versions.
The serial port communication speed must be 115200 baud. 

\subsubsection{usage}

\begin{lstlisting}[mathescape=true]
restinfo.pl [option] $\ldots$ 
\end{lstlisting}

The command line options are:
\begin{description*}
	\item[-c \textless{file}\textgreater] use the specified configuration file
	\item[-h] print help and exit
	\item[-d] run in debugging mode
	\item[-v] print version information and exit
\end{description*}

\subsection{restconfig.pl} \hypertarget{h:restconfig}{}

\cc{restconfig.pl} is used to configure the receiver serial port for 115200 baud and no parity bit.
The latter is necessary for using the SMT 360 with \cc{ntpd}.
The new configuration is written to flash memory.

\subsubsection{usage}

\begin{lstlisting}[mathescape=true]
restconfig.pl [option] $\ldots$ 
\end{lstlisting}

The command line options are:
\begin{description*}
	\item[-c \textless{file}\textgreater] use the specified configuration file
	\item[-h] print help and exit
	\item[-d] run in debugging mode
	\item[-v] print version information and exit
\end{description*}

The serial port device name is read from \cc{gpscv.conf} or the specified file. 

\subsection{restplayer.pl} \hypertarget{h:restplayer}{}

This is used to replay raw data files through a serial port, simulating the 
operation of the GNSS receiver. This is useful for testing the operation of 
\cc{ntpd}, for example. This script will have to be modified for individual 
use because it currently hard codes paths and device names. 
