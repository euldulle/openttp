\section{gziplogs.pl \label{s:gziplogs}}

\cc{gziplogs.pl} is used to manage compression of log files. 
Typically, it will be run once per day, after UTC0. It requires 
a configuration file, \cc{gziplogs.conf}, which is expected to be
in the user's \cc{etc} directory. 

\cc{gziplogs.pl} doesn't produce a log file.

\subsection{usage}

\cc{gziplogs.pl} is normally run as a \cc{cron} job. To run it on the command line, use:
\begin{lstlisting}[mathescape=true]
gziplogs.pl [OPTION] $\ldots$
\end{lstlisting}
The command line options are:
\begin{description*}
	\item[-c \textless file\textgreater] use the specified configuration file
	\item[-d]	run in debugging mode
	\item[-h]	print help and exit
	\item[-m \textless MJD\textgreater] compress files for the given MJD
	\item[-v]	print version information and exit
\end{description*}

\subsection{configuration file}

The configuration file has a single entry.

{\bfseries files}\\
This entry defines a comma-separated list of files to compress. Two date specifications, 
delimited by parentheses, are recognized: YYYYMMDD; and MJD.\\
\textit{Example:}
\begin{lstlisting}
files = raw/{MJD}.rx, raw/{MJD}.tic, raw/{YYYYMMDD}.dat
\end{lstlisting}


