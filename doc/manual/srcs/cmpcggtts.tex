\subsection{cmpcggtts.py}

\cc{cmpcggtts.py} matches tracks in paired CGGTTS files. It can be used for time transfer and delay calibration. 

\subsubsection{usage}

\begin{lstlisting}[mathescape=true]
cmpcggtts.py [OPTION] $\ldots$ refDir calDir firstMJD lastMJD
\end{lstlisting}
The command line options are:
\begin{description*}
 \item[-{}-help]            show this help message and exit
 \item[-{}-starttime STARTTIME] start time of day HH:MM:SS for first MJD (default 0)
 \item[-{}-stoptime STOPTIME]   stop time of day HH:MM:SS for last MJD (default 23:59:00)
 \item[-{}-elevationmask ELEVATIONMASK] elevation mask (in degrees, default 0.0)
 \item[-{}-mintracklength MINTRACKLENGTH] minimum track length (in s, default 750)
 \item[-{}-maxdsg MAXDSG]       maximum DSG (in ns, default 20.0)
 \item[-{}-matchephemeris ]     match on ephemeris (default no)
  \item[-{}-cv]                  compare in common view (default)
  \item[-{}-aiv]                 compare in all-in-view
  \item[-{}-acceptdelays]        accept the delays (no prompts in delay calibration mode)
  \item[-{}-delaycal]            delay calibration mode
  \item[-{}-timetransfer]        time-transfer mode (default)
  \item[-{}-ionosphere]          use the ionosphere in delay calibration mode (default = not used)
   \item[-{}-checksum]           exit on bad checksum (default = warn only)
  \item[-{}-refprefix REFPREFIX] file prefix for reference receiver (default = MJD)
  \item[-{}-calprefix CALPREFIX] file prefix for calibration receiver (default = MJD)
  \item[-{}-refext REFEXT]       file extension for reference receiver (default = cctf)
  \item[-{}-calext CALEXT]       file extension for calibration receiver (default = cctf)
  \item[-{}-debug, -d]           debug (to stderr)
  \item[-{}-nowarn]              suppress warnings
  \item[-{}-quiet]               suppress all output to the terminal
  \item[-{}-keepall]             keep tracks after the end of the day
  \item[-{}-version, -v]         show version and exit
\end{description*}

\subsubsection{examples}

Basic common-view time transfer with CGGTTS files in the folders \cc{refrx} and \cc{remrx}:
\begin{lstlisting}[mathescape=true]
cmpcggtts.py refrx remrx 57555 57556
\end{lstlisting}

Delay calibration with filenames according to the CGGTTS V2E specification:
\begin{lstlisting}[mathescape=true]
cmpcggtts.py --delaycal --refprefix GZAU01 --calprefix GZAU99 refrx remrx 57555 57556
\end{lstlisting}

\subsubsection{}

The default is time-transfer. In this mode, a linear fit to the time-transfer data is made and the 
fractional frequency error and REFSYS (evaluated at the midpoint of the data set) are outputted. 
The uncertainties are estimated from the linear fit.

In delay calibration mode, the presumption is that the data are for two receivers sharing a
common clock and operating on a short baseline. The ionosphere correction is removed by default
but can be retained via a command line option. Delays as reported in the CGGTTS header can be 
changed interactively; prompting for the new delays can be skipped with a command line option.


Data can be filtered by elevation, track length and DSG.

Matching on ephemeris (IODE) can reduce time-transfer noise.

Three text files are produced:




