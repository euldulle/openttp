\section{Javad/Topcon receivers}

\subsection{jnslog.pl}
\hypertarget{h:jnslog}{}

\cc{jnslog.pl} is used to configure and log Javad 
GPS Receiver Interface Language (GRIL) receivers.

It doesn't produce a log file.

GRIL receivers are obsolete and further development of the software
described here has ceased.

\subsubsection{usage}

\begin{lstlisting}[mathescape=true]
jnslog.pl [option] $\ldots$ configurationFile
\end{lstlisting}

The command line options are:
\begin{description*}
 \item[-h] print help and exit
 \item[-d] run in debugging mode
 \item[-r] suppress the reset of the receiver on startup (NOTE! automatic reset)
 \item[-v] print version information and exit
\end{description*}

\subsubsection{configuration}

The file \cc{receiver.conf} lists the commands used to configure the receiver.
For example, to configure a single frequency receiver:
\begin{lstlisting}
		# GRIL commands
    set,out/elm/dev/ser/a,$Init{receiver:elevation mask}
    set,pos/iono,off  
    set,pos/tropo,off 
    set,dev/pps/a/time,gps
    set,dev/pps/a/offs/ns,$Init{receiver:pps offset}
    set,dev/pps/a/per/ms,1000
    set,dev/pps/a/out,on
    
    # Receiver messages to turn on
    RT                  # Receiver time
    TO                  # Receiver base time to receiver time offset
    ZA                  # PPS offset (sawtooth)
    YA                  # Extra time offset 
    SI                  # Satellite index
    EL                  # Satellite elevation
    AZ                  # Satellite azimuth
    SS                  # Satellite navigation status
    FC                  # C/A lock loop status bits
    F1 
    F2 
    RC                  # Full pseudorange C/A
    R1                  # Full pseudorange P/L1
    R2                  # Full pseudorange P/L2
    P1                  # For RINEX Obs
    P2                  # For RINEX Obs
    RD                  # For RINEX
    NP                  # Navigation Position text message
    UO:{3600,14400,0,2} # UTC parameters, when changes or every 4 hrs
    IO:{3600,14400,0,2} # Ionospheric parameters
    GE:{1,3600,0,2}     # GPS ephemeris data
\end{lstlisting}
The GRIL commands in the first part of the file are executed verbatim, with subsitution
of values from the configuration file, written as Perl hash table lookups. 
Messages to be enabled are then listed, one per line. The message rate can be specified 
in the GRIL syntax.

Configuration commands in \cc{gpscv.conf} specific to this receiver are:

JNS receivers have a number of specific configuration entries in \cc{gpscv.conf}:
\begin{itemize}
\item \hyperlink{h:configuration}{receiver::configuration}
\item \hyperlink{h:pps_synchronization}{receiver::pps synchronization}
\item \hyperlink{h:pps_synchronization_delay}{receiver::pps synchronization delay}
\item \hyperlink{h:paths_rinex_l1l2}{paths::rinex l1l2}
\end{itemize}

\subsection{jnsextract.pl}
\hypertarget{h:jnsextract}{}

\cc{jnsextract.pl} is used to decode and extract information from Javda receiver log files. 
It will decompress the file if necessary.
It doesn't use \cc{gpscv.conf}. 

\subsubsection{usage}

\begin{lstlisting}[mathescape=true]
jnsextract.pl [option] $\ldots$ file
\end{lstlisting}

The command line options are:
\begin{description*}

	\item[-b \textless{value}\textgreater]  start time (s or hh:mm, default 0)
	\item[-c] compress (eg skip repeat values) if possible
	\item[-e \textless{value}\textgreater]  stop time (s or hh:mm, default 24:00)
	\item[-g] select GPS only
	\item[-i \textless{n}\textgreater]  decimation interval (in seconds, default=1)
	\item[-k]  keep the uncompressed file, if created
	\item[-o \textless{mode}\textgreater] output mode:
		\begin{description*}
			\item[bp]  receiver sync to reference time (BP message)
			\item[cn]  carrier-to-noise vs elevation
			\item[do]  DO derivative of time offset vs time
			\item[si]  visibility vs time (SI message)
			\item[np]  visibility vs time (NP message)
			\item[vt]  visibility vs time (list time for each SV)
			\item[tv]  visibility vs time (list SV for each time)
			\item[tr]  satellite tracks (PRN, azimuth, elevation)
			\item[tn]  satellite tracks (PRN, azimuth, elevation, CN)
			\item[to]  TO time offset vs time
			\item[ya]  YA time offset vs time
			\item[za]  ZA time offset vs time
			\item[st]  ST solution time tag vs time
			\item[uo]  UO UTC parameters
		\end{description*}
	\item[-r]    select GLONASS only

\end{description*}

\subsection{runrinexobstc.pl}

\hypertarget{h:runrinexobstc}{}

\cc{runrinexobstc.pl} runs the now deprecated \cc{rinexobstc}, which produces a
RINEX observation file suitable for precise positioning when used with a 
suitable Javad receiver. \cc{mktimetx} can now be configured to produce suitable output.

\cc{runrinexobstc.pl} doesn't produce a log file.
	
\subsubsection{usage}
\cc{runrinexobstc.pl} is normally run as a \cc{cron} job.

To run \cc{runrinexobstc.pl} on the command line, use
\begin{lstlisting}[mathescape=true]
runrinexobstc.pl [option] $\ldots$ [Start MJD  [Stop MJD]]
\end{lstlisting}

\cc{Start MJD} and \cc{Stop MJD} specify the range of MJDs to process.
If a single MJD is specified, then data for that day is processed. If no
MJD is specified, the previous day's data is processed.

The options are:
\begin{description*}
	\item[-a \textless{file}\textgreater]  extend check for missed processing back \cc{n} days 
		(the default is~7)
	\item[-c \textless{file}\textgreater] use the specified configuration file
	\item[-d]	run in debugging mode
	\item[-h]	print help and exit
	\item[-x] run missed processing
	\item[-v]	print version information and exit
\end{description*}

\subsubsection{configuration}

\cc{runrinexobstc.pl} uses \cc{gpscv.conf}. 

It uses the following specific configuration file entries in \cc{gpscv.conf}; most
relate to writing the RINEX observation file header:
\begin{itemize}
	\item \hyperlink{h:antenna_x}{antenna::x}
	\item \hyperlink{h:antenna_y}{antenna::y}
	\item \hyperlink{h:antenna_z}{antenna::z}
	\item \hyperlink{h:antenna_marker_name}{antenna::marker name}
	\item \hyperlink{h:antenna_marker_number}{antenna::marker number}
	\item \hyperlink{h:antenna_delta_e}{antenna::delta e}
	\item \hyperlink{h:antenna_delta_h}{antenna::delta h}
	\item \hyperlink{h:antenna_delta_n}{antenna::delta n}
	\item \hyperlink{h:antenna_antenna_number}{antenna::antenna number}
	\item \hyperlink{h:antenna_antenna_type}{antenna::antenna type}
	\item \hyperlink{h:rinex_agency}{rinex::agency}
	\item \hyperlink{h:rinex_observer}{rinex::observer}
	\item \hyperlink{h:paths_rinex_l1l2}{paths::rinex l1l2}
\end{itemize}
