\section{NVS NV08C receivers}

\subsection{nv08log.pl} \hypertarget{h:nvslog}{}

\cc{nvo8log.pl} is used to configure and log NVS NV08 receivers.
It needs the Perl library \cc{NV08C}.

It doesn't produce a log file.

There are no NV08-specific configuration commands in \cc{gpscv.conf}

\subsubsection{usage}

\begin{lstlisting}[mathescape=true]
nv08log.pl [option] $\ldots$ 
\end{lstlisting}

The command line options are:
\begin{description*}
	\item[-c \textless{file}\textgreater] use the specified configuration file
	\item[-h] print help and exit
	\item[-d] run in debugging mode
	\item[-r] reset the receiver on startup
	\item[-v] print version information and exit
\end{description*}

\subsection{nv08extract.pl} \hypertarget{h:nv08extract}{}

\cc{nv08extract.pl} is used to decode and extract information from NVS NV08 receiver log files. 
It will decompress the file if necessary.

It uses \cc{gpscv.conf} to construct receiver log file names if the file is not explicitly given.

\subsubsection{usage}

\begin{lstlisting}[mathescape=true]
nv08extract.pl [option] $\ldots$ [file]
\end{lstlisting}

Given an MJD via a command line option, it will construct a file name using the information in \cc{gpscv.conf}. If no MJD or file is given, it assumes the current MJD for the file. 

The command line options are:
\begin{description*}
	\item[-c \textless{file}\textgreater] use the specified configuration file
	\item[-h] print help and exit
	\item[-d] run in debugging mode
	\item[-v] print version information and exit
	\item[-m \textless{MJD}\textgreater]  MJD of the file to process
	\item[-t] extract Time, Date and Time Zone offset
	\item[-o] extract Receiver Operating Parameters
	\item[-s] extract Visible Satellites
	\item[-n] extract Number of Satellites and Dilution Of Precision (DOP)
	\item[-w] extract Software Version, Device ID and Number of Channels
	\item[-f] extract 1 PPS 'sawtooth' correction
	\item[-T] extract Time Synchronisation Operating Mode (antenna cable delay, averaging time)
	\item[-p] extract PVT Vectors and associated quality factors (including TDOP)
	\item[-a] extract Additional Operating Parameters
	\item[-P] extract Port status messages
	\item[-e] extract Satellite ephemeris
	\item[-l] extract Time scale parameters
	\item[-i] extract Ionosphere parameters
	\item[-g] extract GPS, GLONASS and UTC time scale parameters
	\item[-r] extract Raw data (pseudoranges, etc)
	\item[-G] extract Geocentric antenna coordinates in WGS-84 system
	\item[-u] extract Unknown message (garbage data)
	\item[-z] less verbose output
\end{description*}

\subsection{nv08info.pl} \hypertarget{h:nv08info}{}

This actually does nothing useful. If it did, it would query the receiver for its
serial number and so on. 

\subsubsection{usage}

\begin{lstlisting}[mathescape=true]
nv08info.pl [option] $\ldots$ 
\end{lstlisting}

The command line options are:
\begin{description*}
	\item[-c \textless{file}\textgreater] use the specified configuration file
	\item[-h] print help and exit
	\item[-d] run in debugging mode
	\item[-v] print version information and exit
\end{description*}
