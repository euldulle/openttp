\subsection{fetchigs.py}

\hypertarget{h:fetchigs}{}

\cc{fetchigs.py} uses the Python library \cc{urllib} to download GNSS products and station observation
files from IGS data centres. You will need a configuration file that sets up downloads from at least one IGS data
centre. The sample configuration file should be sufficient for most uses.

\subsubsection{usage}

\begin{lstlisting}[mathescape=true]
fetchigs.py [OPTION] $\ldots$ <start> [<stop>]
\end{lstlisting}

The \cc{start} and \cc{stop} times can be in the format:
\begin{description*}
\item[MJD] Modified Julian Date
\item[yyyy-doy] year and day of year (1 \textellipsis)
\item[yyyy-mm-dd] year, month (1-12) and day (1 \textellipsis)
\end{description*}

The command line options are:
\begin{description*}
	\item[-{}-help, -h] print help and exit
	\item[-{}-config \textless CONFIG \textgreater] , -c use this configuration file
	\item[-{}-debug, -d]           print debugging output to stderr
	\item[-{}-outputdir \textless DIR \textgreater] set the output directory
	\item[-{}-ephemeris]           get broadcast ephemeris
	\item[-{}-clocks]              get clock products (.clk)
	\item[-{}-orbits]              get orbit products (.sp3)
	\item[-{}-rapid]               fetch rapid products
	\item[-{}-final]               fetch final products
	\item[-{}-centre \textless CENTRE \textgreater]       set data centres
	\item[-{}-listcentres, -l]     list available data centres
	\item[-{}-observations]        get station observations (only mixed observations for V3)
	\item[-{}-statid \textless STATID \textgreater]      station identifier (eg V3 SYDN00AUS, V2 sydn)
	\item[-{}-rinexversion \textless 2,3 \textgreater ]  rinex version of station observation files
	\item[-{}-system \textless SYSTEM \textgreater]      gnss system (GLONASS,BEIDOU,GPS,GALILEO,MIXED
	\item[-{}-noproxy]             disable use of proxy server
	\item[-{}-proxy \textless PROXY \textgreater]    set your proxy server (server:port)
	\item[-{}-version, -v ]       print version information and exit
\end{description*}

\subsubsection{examples}

This downloads V3 mixed observation files from the IGS station CEDU, with the identifier CEDU00AUS
for days 10 to 12 in 2020.\\
\begin{lstlisting}
fetchigs.py --centre CDDIS --observations --statid CEDU00AUS --version 3 --system MIXED 2010-10 2020-12
\end{lstlisting}

This downloads final IGS clock and orbit products for MJD 58606 from the CDDIS data centre.\\
\begin{lstlisting}
fetchigs.py --centre CDDIS --clocks --orbits --final 58606
\end{lstlisting}

This downloads a \cc{brdc} broadcast ephemeris file.\\
\begin{lstlisting}
fetchigs.py --centre GSSC --ephemeris --system MIXED --rinexversion 2 58606
\end{lstlisting}

\subsubsection{configuration file}

The \cc{[Main]} section has two keys.

{\bfseries Data centres}\\
This is a comma-separated list of sections which define IGS data centres which can be used to download data.\\
\textit{Example:}
\begin{lstlisting}
Data centres = CDDIS,GSSC
\end{lstlisting}

{\bfseries Proxy server}\\
This sets a proxy server (and port) to be used for downloads.\\
\textit{Example:}
\begin{lstlisting}
Proxy server = someproxy.in.megacorp.com:8080
\end{lstlisting}

Each defined IGS data centre has the following keys, defining various paths.

{\bfseries Base URL}\\
This sets the base URL for downloading files.\\
\textit{Example:}
\begin{lstlisting}
Base URL = ftp://cddis.gsfc.nasa.gov
\end{lstlisting}

{\bfseries Broadcast ephemeris}\\
This sets the path relative to the base URL for downloading broadcast ephemeris files.\\
\textit{Example:}
\begin{lstlisting}
Broadcast ephemeris = gnss/data/daily
\end{lstlisting}

{\bfseries Products}\\
This sets the path relative to the base URL for downloading IGS products.\\
\textit{Example:}
\begin{lstlisting}
Products = gnss/products
\end{lstlisting}

{\bfseries Station data}\\
This sets the path relative to the base URL for downloading IGS station RINEX files. \\
\textit{Example:}
\begin{lstlisting}
Station Data = gnss/data/daily
\end{lstlisting}
